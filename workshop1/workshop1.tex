\documentclass{article}
\usepackage{tabularx}
\usepackage{amsmath}
\usepackage{setspace}
\usepackage{color}
\usepackage{tikz}
\usepackage{graphicx}
\usepackage[margin=1in]{geometry}
\usepackage{xepersian}
\settextfont{Yas}
% Fixture for Xepersian 23 bug of setting persian math digit fonts
\ExplSyntaxOn \cs_set_eq:NN \etex_iffontchar:D \tex_iffontchar:D \ExplSyntaxOff
\setmathdigitfont{Yas}
\onehalfspacing
\renewcommand{\labelenumii}{\alph{enumii})}
\let\oldlabelenumi=\labelenumi
\let\oldlabelenumii=\labelenumii
\newcommand*\circled[1]{
	\tikz[baseline=(char.base)]{
		\node[shape=circle,draw,inner sep=-1pt,color=red] (char) {\textcolor{black}{#1}\strut}
	}\kern-3pt
}
% Enumi
\newcommand{\StartCircledEnumi}{\renewcommand{\labelenumi}{\circled{\oldlabelenumi}}}
\newcommand{\EndCircledEnumi}{\renewcommand{\labelenumi}{\oldlabelenumi}}
% Enumii
\newcommand{\StartCircledEnumii}{\renewcommand{\labelenumii}{\circled{\oldlabelenumii}}}
\newcommand{\EndCircledEnumii}{\renewcommand{\labelenumii}{\oldlabelenumii}}
\begin{document}
	\begin{center}
		\Huge
		مبانی نظریه محاسبه
	\end{center}
	\Large
	\begin{tabularx}{\linewidth}{>{\raggedleft\arraybackslash}X}
		کارگاه اول
		\\
		مبحث: زبان‌ها، تعاریف بازگشتی و استقرای ساختاری
		\\
		\textcolor{red}{توجه:}
		سوالات علامت دار برای تمرین بیشتر شما قرار داده شده‌‌اند. تدریسیار در صورت داشتن فرصت کافی می‌تواند آن‌ها را در کارگاه بررسی کند.
	\end{tabularx}
	\rule{\textwidth}{1pt}
	\begin{enumerate}
		\item
		برای هر زبان $L$ اگر داشته باشیم
		$L^2 \subseteq L$
		نشان دهید
		$LL^* \subseteq L$.
		\item
		برای هر زبان $L_1$ و $L_2$ نشان دهید
		$(L_{1}^{*}L_{2}^{*}L_{1}^{*})^* = (L_1 \cup L_2)^*$.
			\item 
			برای زبان‌های زیر یک تعریف بازگشتی ارائه دهید و درستی آن را اثبات کنید.
			\begin{enumerate}
				\item 
				$\{a^ib^j \; | \; i \geq j \}$
				\item 
				$\{a^ib^j \; | \; i \leq j \leq 2i \}$
				%\StartCircledEnumii
				%\item 
				%\EndCircledEnumii
				%$\{a^ib^j \; | \; i \leq j \}$
				%\StartCircledEnumii
				%\item 
				%\EndCircledEnumii
				%$\{a^ib^j \; | \; j \geq 2i\}$
				%\StartCircledEnumii
				%\item 
				%\EndCircledEnumii
				%$\{a^ib^j \; | \; j \leq 2i\}$
			\end{enumerate}
		\item
		برای زبان معادل هر یک از تعاریف بازگشتی زیر، یک تعریف غیر بازگشتی ارائه دهید و درستی آن را اثبات کنید.
		
		\begin{enumerate}
			\item 
			$a \in L$
			و
			$\forall x \in L :  xb, xba \in L$
			\item
			$\Lambda \in L$
			و
			$\forall x,y \in L : axb, bxa, xy \in L$
			\item 
			$\Lambda \in L$
			و
			$\forall x,y \in L :  axby, bxay \in L$
			\StartCircledEnumii
			\item 
			\EndCircledEnumii
			$\Lambda \in L$
			و
			$\forall x \in L : ax, axb \in L$
			\StartCircledEnumii
			\item 
			\EndCircledEnumii
			$a \in L$
			و
			$\forall x \in L :  xb, ax, bx \in L$

			\StartCircledEnumii
			\item
			\EndCircledEnumii
			$\Lambda \in L$
			و
			$\forall x \in L : xa, xba \in L$

		\end{enumerate}
	\item
فرض کنید 
$\Sigma = \{a, b\}$
و
$x,y \in \Sigma^*$
و همچنین
$x,y \neq \Lambda$.
اگر 
$xy = yx$
نشان دهید 
$z \in \Sigma^* $ 
و اعداد طبیعی $i$ و $j$ وجود دارند به‌طوری که
$x = z^i$
و
$y = z^j$.	


	\item
	فرض کنید 
	$L_1, L_2, L_3 \subseteq \Sigma^*$
	زبان‌هایی روی الفبای $\Sigma$ باشند. برای هر یک از موارد زیر رابطه بین دو زبان داده شده را مشخص کنید. آیا همواره با یک دیگر برابرند؟ در غیر این صورت آیا یکی همواره زیر مجموعه دیگری است؟ برای پاسخ خود دلیل بیاورید؛ یعنی اثبات کنید یا مثال نقض ارائه دهید.
	\begin{enumerate}
		\item 
		$L_1(L_2 \cap L_3)$
		و
		$L_1L_2 \cap L_1L_3$
		\item
		$L_1^* \cap L_2^*$
		و
		$(L_1 \cap L_2)^*$
		\item
		$L_1^*L_2^*$
		و
		$(L_1L_2)^*$
		\StartCircledEnumii
		\item 
		\EndCircledEnumii
		$L_1^* \cup L_2^*$
		و
		$(L_1 \cup L_2)^*$
	\end{enumerate}

\StartCircledEnumi
\item
\EndCircledEnumi
فرض کنید 
$\Sigma = \{a, b\}$.
زبان 
$L \subseteq \Sigma^*$
به صورت زیر تعریف شده است:
$$a \in L; \; \forall x,y \in L : ax, bxy, xby, xyb \in L$$
نشان دهید
$L = \{x \in \Sigma^* \; | \; n_a(x) > n_b(x)\}$.

\StartCircledEnumi
\item
\EndCircledEnumi
فرض کنید
$L = \{yy \; | \; y \in \{a, b\}^*\}$. 
آیا دو زبان غیر بدیهی $A$ و $B$ وجود دارند که $L=AB$؟ در صورت وجود مثال بزنید و در غیر این صورت اثبات کنید.
(زبان $X$ غیر بدیهی است اگر $X \neq L, \{\Lambda\}$.)
\StartCircledEnumi
\item
\EndCircledEnumi
فرض کنید 
$L \subseteq \{a, b\}^*$
زبان تمام رشته‌هایی باشد که شامل زیر رشته $bb$ نباشند و همچنین به $b$ ختم نشوند. زبان متناهی $S$ را پیدا کنید به‌طوری که $L = S^*$. در صورتی که شرط دوم (ختم نشدن به $b$) را برداریم نشان دهید هیچ زبانی (خواه متناهی یا نامتناهی) مانند $S$ وجود ندارد که 
$L = S^*$.
\StartCircledEnumi
\item
\EndCircledEnumi
درجه زبان $L$  یا
$\deg(L)$
برابر کوچکترین $k$ است که
$\bigcup\limits_{i = 0}^{k} L^i = L^*$. 
اگر چنین عددی موجود نباشد تعریف می‌کنیم
$\deg(L) = -1$.
\begin{enumerate}
	\item 
	زبانی را مثال بزنید که درجه آن $3$ باشد.
	\item
	زبان نامتناهی‌ای را مثال بزنید که درجه آن 
	$-1$
	باشد.
	\item
	نشان دهید
	$\deg(\{a\}^* \cup \{b\}^*) = -1$.
	\item
	نشان دهید برای هر $n$ زبان $L$ وجود دارد که 
	$\deg(L) = n$.
	
\end{enumerate}
	\end{enumerate}
\begin{flushleft}
	موفق باشید.
\end{flushleft}
	

\end{document}