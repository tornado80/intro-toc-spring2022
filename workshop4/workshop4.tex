\documentclass{article}
\usepackage{tabularx}
\usepackage{amsmath}
\usepackage{amssymb}
\usepackage{setspace}
\usepackage{color}
\usepackage{tikz}
\usepackage{graphicx}
\usepackage{tikz}
\usetikzlibrary{automata}
\usetikzlibrary{positioning}
\usetikzlibrary{arrows}
\tikzset{	node distance=2.5cm, 
	every state/.style={ 
		semithick,
		fill=gray!10},
	initial text={},     
	double distance=2pt, 
	every edge/.style={  
		draw,
		->, %>=stealth’,     
		auto,
		semithick}
}
\let\epsilon\varepsilon
\usepackage[margin=1in]{geometry}
\usepackage{xepersian}
\settextfont{Yas}
% Fixture for Xepersian 23 bug of setting persian math digit fonts
\ExplSyntaxOn \cs_set_eq:NN \etex_iffontchar:D \tex_iffontchar:D \ExplSyntaxOff
\setmathdigitfont{Yas}
\onehalfspacing
\newcommand*\circled[1]{
	\tikz[baseline=(char.base)]{
		\node[shape=circle,draw,inner sep=-1pt,color=red] (char) {\textcolor{black}{#1}\strut}
	}\kern-3pt
}
% Setup Enum
\renewcommand{\labelenumii}{\alph{enumii})}
\let\oldlabelenumi=\labelenumi
\let\oldlabelenumii=\labelenumii
% Enumi
\newcommand{\StartCircledEnumi}{\renewcommand{\labelenumi}{\circled{\oldlabelenumi}}}
\newcommand{\EndCircledEnumi}{\renewcommand{\labelenumi}{\oldlabelenumi}}
% Enumii
\newcommand{\StartCircledEnumii}{\renewcommand{\labelenumii}{\circled{\oldlabelenumii}}}
\newcommand{\EndCircledEnumii}{\renewcommand{\labelenumii}{\oldlabelenumii}}
\newcommand{\CircledEnumi}{\StartCircledEnumi\item\EndCircledEnumi}
\newcommand{\CircledEnumii}{\StartCircledEnumii\item\EndCircledEnumii}
\begin{document}
	\begin{center}
		\Huge
		مبانی نظریه محاسبه
	\end{center}
	\Large
	\begin{tabularx}{\linewidth}{>{\raggedleft\arraybackslash}X}
		کارگاه چهارم
		\\
		مبحث: لِم پامپینگ و مسائل تصمیم
		\\
		
	\end{tabularx}
	\rule{\textwidth}{1pt}
	\large
	\begin{enumerate}
		\item 
		به کمک لِم پامپینگ نشان دهید برای زبان $ L_1 = \{a^ib^ja^k \; | \; k >  i + j \} $ اتوماتای متناهی قطعی وجود ندارند. همچنین به کمک \lr{$L$-distinguishability} نشان دهید برای زبان $ L_2 = \{ww \; | \;  w \in \{a,b\}^*\} $ اتوماتای متناهی قطعی وجود ندارد.
		
		\item 
	برای هر یک از مسائل تصمیم گیری زیر یک الگوریتم ارائه دهید. 
	\begin{enumerate}
		\item 
		با ورودی گرفتن اتوماتای متناهی قطعی $M$ و رشته $x$ مشخص کنید آیا $x$ پیشوند رشته‌ای در $ L(M) $ است؟
		\item
		با ورودی گرفتن اتوماتای متناهی قطعی $M$ و رشته $x$ مشخص کنید آیا $ y \in L(M) $ وجود دارد که $x$ زیررشته‌ی $y$ باشد؟
		\CircledEnumii
		با ورودی گرفتن اتوماتای متناهی قطعی $M$ و دو رشته $x$ و $y$ مشخص کنید آیا این دو رشته نسبت به $ L(M)‌ $ تمییز \LTRfootnote{distinguishable} پذیرند؟
		\CircledEnumii
		با ورودی گرفتن اتوماتای متناهی قطعی $M$ و رشته $x$ مشخص کنید آیا $ x $ پسوند رشته‌ای در $ L(M) $ است؟
		\CircledEnumii
		با ورودی گرفتن دو اتوماتای متناهی $ M_1 $ و $ M_2 $ مشخص کنید آیا $ L(M_1) $ زیر مجموعه $ L(M_2) $ است؟
		\CircledEnumii
		با ورودی گرفتن اتوماتای متناهی 
		$M = (Q, \Sigma, q_0, A, \delta)$
		 و حالت 
		 $q \in Q$
		 مشخص کنید آیا رشته $x$ وجود دارد که $\delta^*(q, x) = q$؟
		\CircledEnumii
		با ورودی گرفتن دو اتوماتای متناهی 
		$ M_1 $
		و
		$ M_2 $
		مشخص کنید آیا برای هر $x \in L(M_1)$، $y \in L(M_2)$ وجود دارد که $x$ پیشوند $y$ باشد؟
	\end{enumerate}
		
	\item
	زبان
	$L \subseteq \{a,b\}^*$
	را پیدا کنید که هیچ \lr{DFA}ای برای آن وجود نداشته باشد ولی $LL$ توسط \lr{DFA}ای پذیرفته شود.
	\begin{latin}
		\item
		Show that for an arbitrary language $A$ which is accepted by a FA, the language $B = \{w \in A \; | \; \text{no prefix of $w$ is member of $A$}\}$ is also accepted by a FA.
		\CircledEnumi
		A set $S$ of nonnegative integers is an \textit{arithmetic progression} if for some integers $n$ and $p$, $S = \{n + ip \; | \; i \geq 0\}$. Let $A \subseteq \{a\}^*$ and $T = \{|x| \; | \; x \in A\}$. 
		\begin{enumerate}
			\item 
			Show that if $T$ is an arithmetic progression, then $A$ can be accepted by an FA.
			\item 
			Conversely, show that if $A$ can be accepted by an FA, then $T$ is the union of a finite number of arithmetic progressions.
		\end{enumerate} 
	\end{latin}
	
	
	
\end{enumerate}
\end{document}