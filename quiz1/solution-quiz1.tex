\documentclass{article}
\usepackage{tabularx}
\usepackage{setspace}
\usepackage{graphicx}
\usepackage{amsmath}
\usepackage{xcolor}
\usepackage{amssymb}
\usepackage[margin=1in]{geometry}
\usepackage{xepersian}
\settextfont{Yas}
% Fixture for Xepersian 23 bug of setting persian math digit fonts
\ExplSyntaxOn \cs_set_eq:NN \etex_iffontchar:D \tex_iffontchar:D \ExplSyntaxOff
\setmathdigitfont{Yas}
\onehalfspacing
\renewcommand{\labelenumii}{\alph{enumii})}
\begin{document}
	\begin{center}
		\Huge
		مبانی نظریه محاسبه
	\end{center}
	\Large
	\begin{tabularx}{\linewidth}{>{\raggedleft\arraybackslash}X>{\raggedright\arraybackslash}X}
		پاسخ کوییز اول
		&
		نمره کل: 9 نمره
	\end{tabularx}
	\rule{\textwidth}{1pt}
	\begin{enumerate}
		\item 
		یک مسئله مثال بزنید و سپس یک زبان را روی آن مسئله شرح دهید.
		\textcolor{cyan}{
		(3 نمره)
		}
	\footnote{
معرفی مسئله، ورودی و خروجی آن یک نمره، مشخص کردن زبان مسئله یک نمره و مشخص کردن الفبای زبان نیز یک نمره دارد.	
}
		\\
		\textbf{پاسخ:}
		یک مسئله تصمیم گیری 
		\LTRfootnote{Decision problem}
		مثال می‌زنیم. فرض کنید بخواهیم با ورودی گرفتن دو عدد طبیعی $n$ و $k$، مشخص کنیم $n$ بر $k$ بخش پذیر است یا خیر. مسئله پارامتر ثابت ندارد و ورودی آن دو عدد $n$ و $k$ و خروجی آن 
		\lr{YES}
		یا
		\lr{NO}
		است. زبان این مسئله را می‌توان به صورت زیر توصیف کرد:
		$$L = \{\langle n,k\rangle  :  n,k \in \mathbb{N}, k \mid n\}$$
		منظور از 
		$\langle n,k\rangle$
		نوعی کد کردن 
		\LTRfootnote{Encoding}
		دو عدد طبیعی با یک الفبا است. به عنوان مثال یک نوع کد کردن می‌تواند به کمک الفبای 
		$\Sigma = \{a, \#\}$
		باشد؛ یعنی دو عدد طبیعی $n$ و $k$ را به صورت
		$a^n\#a^k$
		کد کنیم. در این صورت زبان $L$ به صورت زیر خواهد بود:
		$$L = \{a^n\#a^k  :  n,k \in \mathbb{N}, k \mid n\}.$$
		یک نوع دیگر می‌توانست به کمک الفبای
			$\Sigma = \{0, 1, \#\}$
			باشد و از نمایش مبنای دو اعداد $n$ و $k$ استفاده کنیم. در فصل ۷ خواهید دید که چگونه این نمایش‌ها برای ورودی دادن به ماشین‌های تورینگ---که مدلی ریاضی برای کامپیوتر در دستتان هستند---استفاده می‌شوند.
		\item 
		مسئله معادل اجتماع دو زبان را چگونه توصیف می‌کنید؟ یک مثال بزنید.
				\textcolor{cyan}{
			(۱ نمره)
		}
				\\
		\textbf{پاسخ:}
		مسئله معادل اجتماع دو زبان را می‌توان با قرار دادن کلمه «یا» بین آن دو مسئله توصیف کرد. به عنوان مثال مسئله معادل اجتماع دو زبان
		$L_1 = \{a^{3k+1} \, | \, k \in \mathbb{N} \}$
		و
				$L_2 = \{a^{3k+2} \, | \, k \in \mathbb{N} \}$
				را می‌توان به این صورت توصیف کرد که با ورودی گرفتن یک عدد طبیعی مانند $n$، مشخص کنیم باقی مانده $n$ بر ۳ برابر ۱ 
				\textit{یا}
				۲ است یا خیر.
		
		\item 
		درستی یا نادرستی گزاره‌های زیر را مشخص کنید. برای پاسخ خود دلیل بیاورید؛ در صورت درستی اثبات و در غیر این صورت مثال نقض بیاورید. 
		\textcolor{cyan}{
		(هر مورد ۱ نمره و جمعاً ۵ نمره)
		}
		\begin{enumerate}
			\item 
			برای هر زبان $L$، 
			$L^*$
			نامتناهی است.
			\\
			\textbf{پاسخ:}
			\textcolor{red}{نادرست}.
			اگر
			$ L = \{\Lambda\} $
			باشد، 
			$ L^* = \{\Lambda\} $
			خواهد بود. اگر
			$\Lambda \neq x \in L$
			باشد، گزاره درست خواهد بود.
			\item 
			$\Lambda$
			می‌تواند الفبای یک زبان باشد.
				\\
			\textbf{پاسخ:}
			\textcolor{red}{نادرست}.
			$\Lambda$
			رشته به طول صفر است اما الفبا یک مجموعه متناهی است.
			\item 
			برای زبان $L$، اگر 
			$\Lambda \in L^*$
			باشد، آنگاه
			$\Lambda \in L$.
			\\
						\textbf{پاسخ:}
			\textcolor{red}{نادرست}. 
			اگر 
			$L = \{a\}$
			باشد، آنگاه
			$L^* = \{\Lambda, a, aa, aaa, \cdots\}$.
			\item 
			هیچ زبانی مانند $L$ وجود ندارد که برای هر $k$ داشته باشیم
			$$\bigcup_{i = 1}^{k} L^i = \bigcup_{i = 1}^{\infty} L^i.$$
				\\
			\textbf{پاسخ:}
			\textcolor{red}{نادرست}. 
			زبان‌های
			$L = \Sigma^*$
			یا
			$L = \{\Lambda\}$
 دارای این ویژگی هستند. 
			\item 
			$\{a^*b^*\} = \{a^nb^n \; | \; n \geq 0 \}$.
				\\
			\textbf{پاسخ:}
			\textcolor{red}{نادرست}. 
			$a^3b^4 \in \{a^*b^*\}$
			اما 
			$a^3b^4 \notin \{a^nb^n \; | \; n \geq 0 \}$.
		\end{enumerate}
	\end{enumerate}
\end{document}